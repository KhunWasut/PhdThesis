\begin{spacing}{2.0}
    \section{Summary}

    A molecular dynamics simulation can be interpreted in the probabilistic view, where the Markovian properties hold in the phase space $\Omega$.
    Taking an advantage of the Markovian properties, eigenfunctions and eigenvalues of the transfer operator $\mathcal{T}(\tau)$ contain numerous
    information pertaining the dynamics of any system of interest, where different motions in the system can be viewed as a spectral decomposition
    of the weighted probability density for a particular configuration $u_{t+\tau}(\mathbf{x})$ after a lag time $\tau$. Each of the motion has
    a corresponding eigenvalue, which relates to the relaxation timescale, where each motion decays differently. As each eigenfunction strictly
    corresponds to a specific motion in the system, the terms \textsl{eigenfunction of the transfer operator} and \textsl{reaction coordinate}
    can be use synonymously.

    Time-lagged Independent Component Analysis (TICA) allows us to approximate the eigenfunctions of $\mathcal{T}(\tau)$ under the condition of
    variational principle in a similar fashion to quantum mechanics. The approximated eigenfunctions by TICA can be projected onto the collective
    variable space, allowing us to interpret the physical meaning of different motions in a chemical system. However, if a set of collective
    variable is large, the eigenfunctions obtained from TICA may be hard to decipher. A reduced representation of TICA eigenfunctions through
    Matching Pursuit (MP) can be computed, which reduced the number of variables needed to express the eigenfunctions, allowing us to deduce
    the physical interpretations of each reaction coordinate better.

    After the $m-1$ slowest processes are determined from TICA, a discretization of the input data can be performed in the subspace of those
    $m-1$ reaction coordinates to form a discrete Markov State Model of the process, which predicts a better relaxation timescale than TICA.
    The coarse--grained MSM allows us to interpret the transition probabilities between key metastable states in the system. Therefore, the
    contents presented in this chapter encompass the entire workflow of using the Markovian properties of MD simulations to gain the dynamical
    insights to a chemical reaction, which is summarized in figure XX.

    \placeholder{Workflow Figure}
\end{spacing}
