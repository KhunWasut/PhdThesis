\begin{spacing}{2.0}
    \section{Markovianity of MD Simulations}\label{sec:markovian}

    In 1983, Zwanzig proposed that the transition between two metastable configurations in the phase space can be treated as a continuous time 
    random walk problem. \cite{P-JStatPhys-1983-v30-Zwanzig} In order to understand the Markovianity of a MD trajectory, it is necessary to think 
    about MD simulations in terms of the probability density of possible configurations, where the evolution of configurations over time is governed 
    by a generalized classical master equation for the distribution of the waiting times to change from one configuration to another. The short memory 
    approximation of the transition memory kernels from the master equation can then be approximated, and it ignores the possibility of the later 
    return to the same state. Hence, if the system's dynamics is sufficiently complex and metastable states are chosen sensibly, then this implies 
    the memoryless characteristics of the interstate jumps. \cite{P-JStatPhys-1983-v30-Zwanzig}

    As opposed to the typical view of the MD simulations as composed of distinct trajectories, MSM takes an ensemble approach given that the dynamics 
    is ergodic in the phase space $\Omega$, that is, there always exist a connected configuration to the current configuration in the phase space. 
    Therefore, the evolution of the ensembles of the trajectories take a probabilistic approach in MSM. Since we assert that a transition from one 
    configuration to another configuration after a lag time $\tau$ is Markovian; therefore, starting from a configuration $\mathbf{x}$ at time $t$
    where $\mathbf{x} \in \Omega$, the probability that a trajectory starting at $\mathbf{x}$ at time $t$ will be in an infinitestimally small
    region $d\mathbf{y}$ around point $\mathbf{y} \in \Omega$, $p(\mathbf{x}, \mathbf{y}; \tau)$ can be defined as, 
    \cite{P-JChemPhys-2011-v134-Prinz, P-PhysChemChemPhys-2011-v13-Prinz}

    \begin{equation}
        p(\mathbf{x}, \mathbf{y}; \tau)d\mathbf{y} = \mathbb{P}[\mathbf{x}(t+\tau)\in\mathbf{y}+d\mathbf{y}|\mathbf{x}(t)=\mathbf{x}]
        \label{eq:msm-prob-def}
    \end{equation}

    Therefore, equation \ref{eq:msm-prob-def} implies that for a set of configurations $A \subset \Omega$, the following also holds true for all
    configurations $\mathbf{y} \in A$,

    \begin{equation}
        p(\mathbf{x},A;\tau) = \int_{\mathbf{y}\in A}p(\mathbf{x},\mathbf{y};\tau)d\mathbf{y}
    \end{equation}

    Since the phase space $\Omega$ is assumed to be ergodic, the state $\mathbf{x}$ will be visited infinitely often as $t\to\infty$. Hence, a
    unique stationary probability density $\mu(\mathbf{x})$ can be written to represent this fact, and $\mu(\mathbf{x})$ represents the ensemble's
    equilibrium density; for example, for a canonical ensemble ($NVT$ variables are held constant), $\mu(\mathbf{x})$ can be written as,

    \begin{equation}
        \mu(\mathbf{x}) = \frac{e^{-\beta H(\mathbf{x})}}{Z}
    \end{equation}

    \noindent where $Z = \int\exp(-\beta H(\mathbf{x}))d\mathbf{x}$ is the canonical partition function, $\beta = (k_B T)^{-1}$, and $H(\mathbf{x})$
    is a classical Hamiltonian of the system. In order to model reversible reactions, another assumption that the configuration $\mathbf{x}(t)$ 
    is reversible is needed. Therefore, the following detailed balance condition,

    \begin{equation}
        \mu(\mathbf{x})p(\mathbf{x},\mathbf{y};\tau) = \mu(\mathbf{y})p(\mathbf{y},\mathbf{x};\tau)
    \end{equation}

    \noindent as to be satisfied. Considering a probability density of a configuration $p_t(\mathbf{x})$, the transition probability 
    $p(\mathbf{x},\mathbf{x};\tau)$ governs that after some times $\tau$, the probability density of $\mathbf{x}$ at time $t + \tau$is expressed as
    $p_{t+\tau}(\mathbf{x})$. Hence, one could define a propagator $\mathcal{Q}(\tau)$that satisfies the following properties,

    \begin{equation}
        p_{t+\tau}(\mathbf{y}) = \mathcal{Q}(\tau)p_t(\mathbf{y}) = \int_{\mathbf{x}\in\Omega}p(\mathbf{x},\mathbf{y};\tau)p_t(\mathbf{x})d\mathbf{x}
        \label{eq:msm-propagator}
    \end{equation}

    When weighted by the stationary density, another way to look at equation \ref{eq:msm-propagator} is through the transfer operator $\mathcal{T}(\tau)$
    which propagates the weighted probability density. Thus, it must follow that,

    \begin{equation}
        u_{t+\tau}(\mathbf{y}) = \mathcal{T}(\tau)u_t(\mathbf{y}) = \frac{1}{\mu(\mathbf{y})}
            \int_{\mathbf{x}\in\Omega}p(\mathbf{x},\mathbf{y};\tau)\mu(\mathbf{x})u_t(\mathbf{x})d\mathbf{x}
        \label{eq:msm-weighted-propagator}
    \end{equation}

    In general, $\mathcal{T}(\tau)$ is a preferred operator because of the fact that $\mathcal{T}(\tau)$ operates on the weighted probability density. 
    Therefore, it can be said that $\mathcal{T}(\tau)$ has to conform with the following properties,

    \begin{enumerate}
        \item{$\mathcal{T}(\tau)$ fulfills the Chapman—-Kolmogorov equation $u_{t+k\tau}(\mathbf{x}) = \left[\mathcal{T}(\tau)\right]^k u_t(\mathbf{x})$,
              where $\left[\mathcal{T}(\tau)\right]^k$ represents the $k$-th power of the transfer operator matrix.}
        \item{There exist eigenfunctions $\psi_i$ that corresponds to the eigenvalue problem $\mathcal{T}(\tau)\psi_i = \lambda_i\psi_i$, where
              $\lambda_i$ is the corresponding eigenvalue of an eigenfunction $\psi_i$}
        \item{$\psi_i$ relates to the $i$-th eigenfunction of the propagator $\mathcal{Q}(\tau)$ through the stationary distribution; that is, 
              $\mu(\mathbf{x})^{-1}\phi_i(\mathbf{x}) = \psi_i(\mathbf{x})$. Both $\phi_i$ and $\psi_i$ share the same eigenvalue $\lambda_i$.}
        \item{The eigenvalues $\lambda_i$ are real numbers and $\lambda_i \in \left(-1,1\right]$, and the first eigenvalue $\lambda_1$ is always 1 and 
              corresponds to the stationary density $\mu(\mathbf{x})$, and it must follow that $1 > \lambda_2 \geq \lambda_3 \geq \ldots \geq \lambda_n$}
    \end{enumerate}

    Thus, the weighted probability density $u_{t+\tau}(\mathbf{x})$ can be written as a sum of all the eigenfunctions of $\mathcal{T}(\tau)$, which 
    represent the spectral decomposition of the dynamics in our system. All motions in the dynamics are thought of the superimposition of 
    independent motions represent by the $i$-th eigenfunction. Hence, $u_{t+k\tau}(\mathbf{x})$ is now written as,

    \begin{equation}
        u_{t+k\tau}(\mathbf{x}) = \sum_{i=1}^m \lambda_i^k\left<u_t,\psi_i\right>_{\mu}\psi_i(\mathbf{x}) + 
            \sum_{j=m+1}^{\infty} \lambda_j^k\left<u_t,\psi_j\right>_{\mu}\psi_j(\mathbf{x})
        \label{eq:spectral-density-ktau}
    \end{equation}

    The first term of equation \ref{eq:spectral-density-ktau} represents $m$ slowest motions which are deemed significant, while the second term 
    represents all the other fast motions that may be considered irrelevant to the rate--determining process. $\left<u_t,\psi_i\right>_{\mu}$ 
    is simply the inner product between $u_t$ and $\psi_i$ weighted by the stationary density $\mu_{\mathbf{x}}$. Equation \ref{eq:spectral-density-ktau}
    also implies that in the limit where $k\to\infty$, only the first eigenfunction would remain as all other eigenvalues are strictly less than 
    one except the first one, recovering the equilibrium distribution. The patterns of the eigenvalues thus would imply that for any motions where
    $\lambda_i$ is less than 1, all the terms where $i > 1$ would decay over time according to the value of $\lambda_i$, which also determines the 
    implied relaxation timescale of each motion through the following relationship,

    \begin{equation}
        t_i = -\frac{\tau}{\ln\lambda_i}
    \end{equation}

    Equation \ref{eq:spectral-density-ktau} can thus be now written as,

    \begin{equation}
        u_{t+k\tau}(\mathbf{x}) = 1 + \sum_{i=2}^m e^{-\frac{k\tau}{t_i}}\left<u_t,\psi_i\right>_{\mu}\psi_i(\mathbf{x}) +
            \sum_{j=m+1}^{\infty} \lambda_j^k\left<u_t,\psi_j\right>_{\mu}\psi_j(\mathbf{x})
        \label{eq:spectral-density-ktau-2}
    \end{equation}

    The representation of $u_{t+k\tau}(\mathbf{x})$ outlined in equation \ref{eq:spectral-density-ktau} is now separated into three parts; the stationary 
    distribution, the $m-1$ important process that have distinctly different eigenvalues, from which the implied timescale of the process can now 
    be extracted. However, important information can also be found from the first $m-1$ eigenfunctions as well through the projection onto the 
    collective variable space, which we will discuss in the following section.
\end{spacing}
