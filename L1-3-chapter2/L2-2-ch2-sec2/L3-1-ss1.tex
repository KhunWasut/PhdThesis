\section{Theoretical Foundations of MSM}

\subsection{MSM Ansatz}

\begin{spacing}{2.0}
    In 1983, Zwanzig proposed that the transition between two metastable configurations in the phase space can be treated as a continuous time 
    random walk problem. \cite{P-JStatPhys-1983-v30-Zwanzig} In order to understand the Markovianity of a MD trajectory, it is necessary to think 
    about MD simulations in terms of the probability density of possible configurations, where the evolution of configurations over time is governed 
    by a generalized classical master equation for the distribution of the waiting times to change from one configuration to another. The short memory 
    approximation of the transition memory kernels from the master equation can then be approximated, and it ignores the possibility of the later 
    return to the same state. Hence, if the system's dynamics is sufficiently complex and metastable states are chosen sensibly, then this implies 
    the memoryless characteristics of the interstate jumps. \cite{P-JStatPhys-1983-v30-Zwanzig}

    As opposed to the typical view of the MD simulations as composed of distinct trajectories, MSM takes an ensemble approach given that the dynamics 
    is ergodic in the phase space $\Omega$, that is, there always exist a connected configuration to the current configuration in the phase space. 
    Therefore, the evolution of the ensembles of the trajectories take a probabilistic approach in MSM. Since we assert that a transition from one 
    configuration to another configuration after a lag time $\tau$ is Markovian; therefore, starting from a configuration $\mathbf{x}$ at time $t$
    where $\mathbf{x} \in \Omega$, the probability that a trajectory starting at $\mathbf{x}$ at time $t$ will be in an infinitestimally small
    region $d\mathbf{y}$ around point $\mathbf{y} \in \Omega$, $p(\mathbf{x}, \mathbf{y}; \tau)$ can be defined as, 
    \cite{P-JChemPhys-2011-v134-Prinz, P-PhysChemChemPhys-2011-v13-Prinz}

    \begin{equation}
        p(\mathbf{x}, \mathbf{y}; \tau)d\mathbf{y} = \mathbb{P}[\mathbf{x}(t+\tau)\in\mathbf{y}+d\mathbf{y}|\mathbf{x}(t)=\mathbf{x}]
        \label{eq:msm-prob-def}
    \end{equation}

    Therefore, equation \ref{eq:msm-prob-def} implies that for a set of configurations $A \subset \Omega$, the following also holds true for all
    configurations $\mathbf{y} \in A$,

    \begin{equation}
        p(\mathbf{x},A;\tau) = \int_{\mathbf{y}\in A}p(\mathbf{x},\mathbf{y};\tau)d\mathbf{y}
    \end{equation}

    Since the phase space $\Omega$ is assumed to be ergodic, the state $\mathbf{x}$ will be visited infinitely often as $t\to\infty$. Hence, a
    unique stationary probability density $\mu(\mathbf{x})$ can be written to represent this fact, and $\mu(\mathbf{x})$ represents the ensemble's
    equilibrium density; for example, for a canonical ensemble ($NVT$ variables are held constant), $\mu(\mathbf{x})$ can be written as,

    \begin{equation}
        \mu(\mathbf{x}) = \frac{e^{-\beta H(\mathbf{x})}}{Z}
    \end{equation}

    \noindent where $Z = \int\exp(-\beta H(\mathbf{x}))d\mathbf{x}$ is the canonical partition function, $\beta = (k_B T)^{-1}$, and $H(\mathbf{x})$
    is a classical Hamiltonian of the system.
\end{spacing}
