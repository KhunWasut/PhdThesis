\begin{spacing}{2.0}
    \section{Approximating the Eigenfunctions of the Transfer Operator}

    \subsection{Approximation of the Eigenfunctions through Variational Principle}

    The solution of an eigenvalue problem in equation \ref{eq:spectral-density-ktau} gives us $m-1$ eigenvalues and eigenfunctions that best 
    approximates the probability of arriving at an MD configuration at time $\tau$ after the current time. As mentioned in \ref{sec:markovian},
    the information provided by eigenvalues implies the inherent timescale of their corresponding eigenfunctions. However, the eigenfunctions 
    themselves also contain several important dynamical information of the process of interest. We could see that the inner product
    $\left<u_t,\psi_i\right>$ tells us about the projection of the probability density of the configuration at current time; thus, the 
    eigenfunction $\psi_i$ with the highest eigenvalue contains the contributions that give rise to the slowest motion for this process. 
    In a situation where we have $\lambda_2 \gg \lambda_i\,\,\forall i > 2$, it readily implies that the slowest motion is a dominant motion and 
    the other eigenfunctions decay much faster at lag time $\tau$, and $\psi_2$ would be able to be taken to dominate the dynamics.

    Equations \ref{eq:spectral-density-ktau} and \ref{eq:spectral-density-ktau-2} provided a spectral decomposition of all different motions in 
    the system, where the information of each motion can be obtained from the eigenvalue problem $\mathcal{T}(\tau)\psi_i = \lambda_i\psi_i$.
    The implied timescale of each process could also be obtained from the eigenvalues $\lambda_i = e^{-\frac{\tau}{t_i}}$. However, we still do 
    not know how to extract the dynamical information from the eigenfunctions. Nonetheless, equation \ref{eq:spectral-density-ktau} implies
    that for any function that relates to a configuration $\mathbf{x}$ at time $t$, the time-autocorrelation function of an arbitrary function
    $f$ as a function of $\tau$ can be written as,

    \begin{equation}
        \left<f(\mathbf{x}_t) f(\mathbf{x}_{t+\tau})\right>_t = \sum_{i=1}^{\infty} e^{-\frac{\tau}{t_i}} \left<\phi_i, f\right>^2
        \label{eq:autocorrelation-function}
    \end{equation}

    Hence, the time-autocorrelation function of the $i$-th normalized eigenfunction $\psi_i$ can be used to recover the $i$-th eigenvalue due
    to the fact that for a normalized eigenfunction, $\left<\psi_i(\mathbf{x}_t)\psi_i(\mathbf{x}_{t+\tau})\right>^2 = 1$. Thus, given that the 
    eigenfunctions can be computed, the eigenvalues and the implied timescales can be approximated from the time-autocorrelation function as a 
    function of $\tau$, now called a \textsl{lag time}.

    \begin{equation}
        \tilde{\lambda}_i = \left<\psi_i(\mathbf{x}_t\psi_i(\mathbf{x}_{t+\tau}))\right>
    \end{equation}

    Nevertheless, the main question here is how would we know the eigenfunction. It is very likely that we will never know the true form of the 
    eigenfunctions $\psi_i$. However, since the eigenvalues $\lambda_i$ can be modeled from the time-autocorrelation of $\psi_i$, we always know 
    that for any model eigenfunction $\tilde{\psi}_i$,

    \begin{equation}
        \left<\tilde{\psi}_i(\mathbf{x}_t)\tilde{\psi}_i(\mathbf{x}_{t+\tau})\right> \leq e^{-\frac{\tau}{t_i}}
    \end{equation}

    Hence, the variational principle can be applied to find a good approximation of $\tilde{\psi}_i$ given that we could find such a function that 
    gives $e^{-\frac{\tau}{\tilde{t}_i}} \leq e^{-\frac{\tau}{t_i}}$ with a value of $\tilde{t}_i$, the variational approximation of $t_i$, as close 
    as possible to $t_i$. Since the equality between the modeled and the actual implied timescale only holds if $\tilde{\psi}_i = \psi_i$,
    the modeled eigenfunction $\tilde{\psi}_i$ that best approximates the true eigenfunction $\psi_i$ needs to maximize the modeled implied timescale
    $\tilde{t}_i$, or equivalently, maximize the modeled eigenvalue $\tilde{\lambda}_i$. In fact, the principles behind the application of the 
    variational principle to approximate the eigenfunctions of MSM is analogous to that of quantum mechanics, where the wavefunctions are best 
    approximated by minimizing the energy.
\end{spacing}
