\begin{spacing}{2.0}
    \subsection{Matching Pursuit (MP): Reduced Representation for Eigenfunctions}

    If $\tilde{\psi}_i$ is expanded with a non--orthonormal basis, some of the collective variables may not be entirely independent; hence, 
    complicating the situation further by having coefficients in two or more possible collective variables that are not independent, and
    does not help us in achieving our goal to truly reducing the dimensionality expression of the chemical process of interest. Matching Pursuit 
    algorithm (MP), first proposed by Mallat and Zhang \cite{P-IEEETrans-1993-v41-Mallat}, can help us achieve our goal by finding a sparse 
    solution of $\tilde{\psi}_i$ and reassign the coefficients accordingly.

    Let us begin by suppose that there are $m-1$ modeled eigenfunctions $\tilde{\psi}_i$ such that $2 \leq i < m$. These $\tilde{\psi}_i$
    are the most important components that we obtained from solving the TICA problem from the previous section. If we initially build such 
    eigenfunctions from our library of $N_{\xi}$ collective variables, then each $\tilde{\psi}_i$ is simply a linear combination of those
    $N_{\xi}$ variables as expressed in equation \ref{eq:linear-combination-cv}. However, as each $\tilde{\psi}_i$ is a function of $N_{\xi}$
    variables, often times, it is very hard to use these variables altogether to discern the physical meaning of each $\tilde{\psi}_i$.
    Preferably, one would prefer that each $\tilde{\psi}_i$ has a physically meaningful representation with only dominant collective variables 
    represented. The MP algorithm reduced the representation of $\tilde{\psi}_i$ by finding only the productive coefficients that best summarize 
    the behavior of the function based on our data. To better understand the MP algorithm, we would assume an arbitrary function $f(t)$ 
    that can be written as a linear expansion on a basis $\chi = \left\{\chi_1,\chi_2,\ldots,\chi_{k}\right\}$ that could either be orthonormal 
    or not.

    \begin{equation}
        f(t) = \sum_{i=1}^k b_i\chi_i(t)
    \end{equation}

    The inner product of $f$ and a basis function $\chi_k(t)$, $\left<f,\chi_k(t)\right>$, can be computed from the following equation,

    \begin{equation}\begin{aligned}
        \left<f,\chi_k\right> &= \sum_{i=1}^N b_i\left<\chi_i,\chi_k\right> \\
            &= \sum_{i=1}^N b_i c_{ik}^{\chi}(0) \\
            &= \sum_{i=1}^N c_{ki}^{\chi}(0)b_i \\
            &= \left(\mathbf{C}^{\chi}(0)\mathbf{b}\right)_k
        \label{eq:MP-correlation}
    \end{aligned}\end{equation}

    And a residual norm of $f$, $\left\lVert f_{res}^2 \right\rVert$, is computed as follow,

    \begin{equation}\begin{aligned}
        \left\lVert f_{res}^2 \right\rVert &= \left< \sum_{i=1}^N b_i\chi_i(t) \sum_{j=1}^N b_j\chi_j(t) \right> \\
            &= \sum_{i=1}^N \sum_{j=1}^N b_i b_j \left<\chi_i(t)\chi_j(t)\right> \\
            &= \sum_{i=1}^N \sum_{j=1}^N b_i c^{\chi}_{ij}(0) b_j = \mathbf{b}^{\top}\mathbf{C}^{\chi}(0)\mathbf{b}
        \label{eq:MP-norm}
    \end{aligned}\end{equation}

    \placeholder{Algorithm here}

    The algorithm to compute the productive coefficients, $\mathbf{b}^{\ddagger}$ of $f$ is outlined in figure XX using $\left<f,\chi_k\right>$
    and $\left\lVert f_{res}^2 \right\rVert$ computed from equations \ref{eq:MP-correlation} and \ref{eq:MP-norm}. This presents an iterative 
    method that converges to a set tolerance value of $\left\lVert f_{res}^2 \right\rVert$. The result of this algorithm is the reduced 
    representation of $f$, $f^{\ddagger}$, such that,

    \begin{equation}\begin{aligned}
        f^{\ddagger}(t) &= \sum_{i=1}^{N_{reduced}} b_i^{\ddagger}\chi_i(t) \\
            &\approx f(t)
    \end{aligned}\end{equation}

    \noindent where $N_{reduced} < N$. As shown in \ref{ssec:tica}, TICA computes $\tilde{\psi}_i$ according to the variational principle to get 
    a best approximation that is reasonably close to $\psi_i$. Since TICA computes $\tilde{\psi}_i$ as a linear projection onto the basis of 
    collective variables, the MP projection of $\tilde{\psi}_i$, $\psi_i^{\ddagger}$, can be computed in the same fashion as equation
    \ref{eq:linear-combination-cv},

    \begin{equation}\begin{aligned}
        \psi_i^{\ddagger} &= \sum_{k=1}^{N_{reduced}} b_{ik}^{\ddagger}\chi_k^{MF} \\
            &\approx \tilde{\psi}_i
    \end{aligned}\end{equation}

    $\psi_i^{\ddagger}$ is now expressed as a linear combination of $N_{reduced} < N_{\xi}$ variables.  Therefore, a reduced representation of
    $\psi_i$ would allow us to interpret the dynamics of each motion involving only important collective variables that are dominant in each 
    specific $\psi_i$. This way, one could gain important mechanistic insights into the slowest dynamics in any kinds of systems.
\end{spacing}
