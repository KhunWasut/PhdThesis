\begin{spacing}{2.0}
    \subsection{Time-lagged Independent Component Analysis (TICA)}\label{ssec:tica}

    Suppose that a modeled eigenfunction $\tilde{\psi}_i$ can be modeled as a linear combination of an orthonormal basis of the ansatz $\chi_k$,
    the reconstruction of $\tilde{\psi}_i$ as a linear combination of basis functions in the set $\chi = \left\{\chi_1,\chi_2,\ldots,\chi_{N_{\chi}}\right\}$
    would take the following form,

    \begin{equation}
        \tilde{\psi}_i = \sum_{k=1}^{N_{\chi}} b_{ik}\chi_k
    \end{equation}

    % EXPAND MORE HERE!! 
    \noindent where the optimal set of coefficients $b_{ik}$ can be determined from solving the following eigenvalue problem,

    \begin{equation}
        \mathbf{C}^{\chi}(\tau)\mathbf{b}_i = \mathbf{b}_i\tilde{\lambda}_i(\tau)
        \label{eq:tica-eqn-1}
    \end{equation}

    \noindent where $\mathbf{C}^{\chi}(\tau)$ is the autocorrelation matrix of the ansatz functions at time $\tau$ with the following form,

    \begin{equation}\begin{aligned}
        c_{ij}^{\chi}(\tau) &= \left<\chi_i(\mathbf{x}_t)\chi_j(\mathbf{x}_{t+\tau})\right>_t \\
        c_{ij}^{\chi}(0) &= \left<\chi_i(\mathbf{x}_t)\chi_j(\mathbf{x}_{t})\right>_t
    \end{aligned}\end{equation}

    \noindent under the condition of orthonomal ansatz functions $\left<\chi_i,\chi_j\right>_{\mu} = \left<\chi_i(\mathbf{x}_t)
    \chi_j(\mathbf{x}_t)\right> = \delta_{ij}$, where $\delta_{ij}$ is the Kronecker's delta. However, for a non-orthonomal basis of the set $\chi$,
    the basis must be orthonormalized first through generalizations of equation \ref{eq:tica-eqn-1} by using the autocorrelation of the ansatz 
    function at time 0.

    \begin{equation}
        \mathbf{C}^{\chi}(\tau)\mathbf{b}_i = \mathbf{C}^{\chi}(0)\mathbf{b}_i\tilde{\lambda}_i(\tau)
        \label{eq:tica-eqn-2}
    \end{equation}

    Solving equation \ref{eq:tica-eqn-2} yields an optimal set of the coefficients $\mathbf{b}_{i}$ for any non-orthonormal basis set $\chi$.
    In our case, since we would like to determine the optimal set of collective variables for a particular eigenfunction, we would then construct a 
    basis set consisting of as many collective variables as possible then solve equation \ref{eq:tica-eqn-2} for optimal coefficients for each
    collective variable in $\tilde{\psi}_i$. In order to properly obtain the autocorrelation function in the basis set of the collective variables, 
    we need to subtract the mean of each collective variable to create the mean--free input from our data such that,

    \begin{equation}
        \xi_i^{MF}(\mathbf{x}) = \xi_i(\mathbf{x}) - \left<\xi_i(\mathbf{x})\right>_t
    \end{equation}

    Therefore, we can now write an approximation of the eigenfunction $\tilde{\psi}_i$ as a linear combination of the mean--free collective 
    variables $\xi_k^{MF}$,

    \begin{equation}
        \tilde{\psi}_i = \sum_{k=1}^{N_{\xi}} b_{ik}\xi_k^{MF}
        \label{eq:linear-combination-cv}
    \end{equation}

    Consequently, we have demonstrated that $\tilde{\psi}_i$ can be approximated from the time--auto-corre-lation of itself at a lag time $\tau$,
    and $\tilde{\psi}_i$ can be projected onto the collective variable space. According to equation \ref{eq:spectral-density-ktau}, only $m-1$
    eigenfunctions are needed to sufficiently describe the dynamics. Since $\lambda_i$ and $t_i$ are functions of $\tau$, it is appropriate to only
    select the eigenfunctions such that the implied relaxation timescale is greater than $\tau$ in order to take all the slowest motions beyond
    the lag time into an account. Nonetheless, one problem remains --- since $\tilde{\psi}_i$ is a linear combination of all the collective variables
    in the provided basis set, it can be hard to interpret the physical meaning of $\tilde{\psi}_i$ when it involves a lot of collective variables.
    Hence, a reduced representation of $\tilde{\psi}_i$ written in terms of only relevant collective variables is desirable for the physical 
    interpretation purposes. In order to achieve this, we use an algorithm called Matching Pursuit (MP), which we will discuss in the following section.
\end{spacing}
