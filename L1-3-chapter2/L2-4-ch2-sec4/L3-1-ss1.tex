\begin{spacing}{2.0}
    \section{Markov State Model (MSM)}

    \subsection{Discretization of the Input MD Simulation Data in Collective Variable Space}

    The application of TICA to a mean--free input data in the collective variable space results in a set of $m-1$ slowest eigenfunctions $\tilde{\psi}_i$ 
    with corresponding implied relaxation timescale $\tilde{t}_i$ associated with its eigenvalue $\tilde{\lambda}_i$ that is slower than a lag time
    $\tau$ of interest. According to \placeholder{No\'{e} et al., these $\tilde{\psi}_i$ contain enough information to cover the entire kinetic map with
    cumulative kinetic variance very close to 1. [CITE]} However, these MD input data are often very large, containing millions of data points. This
    presents a major challenge in the analysis of this data, because many algorithms scale poorly for matrices with size in the order of millions by
    millions. In order to ease the computational workload to build a kinetic model of an MD trajectory, discretization of a large data set with respect
    to an appropriate subspace that is thought to completely describe the whole data is a viable strategy that sparsifies a large matrix into a more
    managable problem. 
\end{spacing}
