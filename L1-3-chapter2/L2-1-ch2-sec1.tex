\begin{spacing}{2.0}
    \chapter{Identification of Reaction Coordinates and the Kinetic Model of Chemical Reactions}
    \label{chap:chapter-msm}

    \section{Introduction}

    Although intuitions allow us to describe a reaction coordinate that suits the narratives of the researchers' perceptions, the main problem in 
    determining the contributing collective variables for a reaction coordinate from pure intuition is there are no systematic ways to confirm the 
    researchers' beliefs. The fact that $n_+$ was used as a collective variable in the solvent coordinate demonstrates the intuition-based choice 
    that are relatively accepted in many recent literatures. For example, the recent work of Raiteri et al. uses this fact to construct 
    two-dimensional free energy landscapes with interionic separation ($r_{+-}$) and $n_+$ as collective variables of the alkali earth ions pairing 
    interactions with carbonate ions in aqueous solution to validate their empirical potential model for these ions. \cite{P-JPhysChemC-2015-v43-Raiteri} 
    Roy et al. also recently published a two-dimensional free energy calculation using the same set of C.V.'s for alkali ions. 
    \cite{P-JPhysChemC-2016-v120-Roy, P-JCTC-2017-v13-Roy} However, the work of Mullen et al. 
    challenged this belief by claiming that $n_+$  does not play a significant role in the process at all. On the other hand, the solvent coordinate 
    that plays an important role for this process are $n_B$ and $\rho_{ii}$ coordinates mentioned in \ref{sec:L12L22}. \cite{P-JCTC-2014-v10-Mullen}

    The significance of the work of Mullen et al. was that it was the first work that performed a systematic analysis of multiple collective variables 
    to determine the best contribution of the collective variables according to the reaction coordinate. However, we have mentioned in \ref{sec:L12L23} 
    that there are some possible issues that arose from their results, mainly his crude restriction of the reaction coordinate that should only 
    consist of a linear combination of 3 collective variables, while there are possibly more collective variables than 3 that could actually involve 
    in the process. Moreover, their work mainly focused on the SSIP - CIP transition without thoroughly covering the behavior of the bulk region. 
    Therefore, the association pattern from the bulk still remained relatively little understood. Also, the efficiency of the method outlined in 
    Mullen et al. also rely a lot on the fact that we could accurately find two isocommittor surfaces and the events that the simulation trajectory 
    crosses between these two, which could be cumbersome in the case where systems do not have a well-defined barrier or the system with multiple 
    metastable states that are close to one another with relatively low barrier.

    As mentioned in Rohrdanz et al., there are many possible methods that can be used to tackle this problem. \cite{P-AnnuRevPhysChem-2013-v64-Rohrdanz} 
    Out of the methods highlighted in Rohrdanz et al., the Markov State Model (MSM) is a viable candidate yet it is still not being applied to this 
    type of problem. In our opinion, the fact that any MD simulations that is ergodic are shown to exhibit Markovian properties \cite{P-JStatPhys-1983-v30-Zwanzig} 
    implies that we could express the dynamics of any MD simulations in terms of their chatacteristic eigenfunctions and eigenvalues corresponding 
    to specific transitions between metastable states [CITE MSM]. Therefore, MSM allows a characterization of any transition between any metastable 
    states besides the slowest motion across the highest free energy barrier. The MSM eigenfunctions also entail important information on dynamical 
    variables, and could be projected onto the collective variable space to gain valuable insights on all possible variables corresponding to 
    different processes in a chemical system. All of these benefits could be achieved with a relatively short simulation time; hence, MSM is an 
    attractive tool among simulations in biochemical processes [CITE]. Nevertheless, there are no MSM interpretations for processes governing the 
    solution dynamics at all. Therefore, we hope to present the MSM interpretation of two model reactions for ionic association process of 
    \ce{LiCl} and \ce{NaCl} in aqueous solutions.
\end{spacing}
