\chapter{Multidimensional Free Energy Computation}

\section{Importance of Free Energy Computation}

\begin{spacing}{2.0}
    The free energy of a chemical system governs the macroscopic behavior along with its thermal fluctuation including both the energetic and 
    entropic influences, and the free energy landscape may contain one or more local minima representing configurations at the metastable states, 
    which are the locations where the system spends a significant amount of time. The change in the system from the reactant to the product states, 
    therefore, is represented in the free energy landscape as a transition from one metastable state to another. During the course of the reaction, 
    the system needs to pass through the free energy barrier, where the probability of barrier crossing with respect to a metastable state labeled as 
    state 1 is defined as,

    \begin{equation}
        p(1\to *) \propto e^{-\beta\Delta A_{1\to *}}
        \label{eq:free-energy-prob-defn}
    \end{equation}

    Equation \ref{eq:free-energy-prob-defn} can be further rearranged such that for any configuration $\mathbf{x} \in \Omega$,  the relative free 
    energy of the configuration $\mathbf{x}$ is related to the equilibrium probability of finding the configuration $\mathbf{x}$ in the phase space,

    \begin{equation}
        A(\mathbf{x}) = -\beta^{-1}\ln p(\mathbf{x})
    \end{equation}
\end{spacing}
