\begin{spacing}{2.0}
    \section{Validation of GPR Free Energy Surfaces}

    The previous sections have listed all the theoretical requirements for free energy landscape reconstruction for any $D$-dimensional problems 
    using GPR. However, in order to ensure that the GPR free energy landscapes are sound, the errors of the GPR free energy surfaces are needed
    to be computed quantitatively. In fact, as part of the characteristic joint distribution between vectors $\mathbf{f}'$ and $\mathbf{f}$ in
    equation XX, it is also possible to compute the variance of all the points in the test data. \cite{W-Stanford-GPR} However, since the 
    computation of variance involves the $(\Sigma')^2 I$ matrix of the training data's variance, it is not a representative variance of the 
    actual free energy landscape computed from a more traditional method with minimum statistical noises of the free energy estimators like US
    or TI. 

    In order to validate the GPR free energy landscape, GPR results can be compared with a reference surface computed using methods that are
    more widely accepted among the community, which implies that the best choice of a reference surface must come from windowed simulations. 
    Although using a reference surface from windowed simulation is not usually a problem in a one-dimensional case, it becomes more problematic
    for the two-dimensional space and beyond due to the scaling issue, which defeats the main purpose of needing a method for free energy
    computation that is fast and resource-efficient. 

    As Mones et al. have highlighted that the one-dimensional windowed simulations are computationally cheap enough to perform, one could easily
    get $D$ one-dimensional free energy landscapes for each individual dimension in our problem. A simple thought experiment would validate that
    it is more beneficial to perform $D$ one-dimensional windowed simulation than to perform one $D$-dimensional windowed simulation if $D > 1$.
    Let $N_{1D}$ be the number of windows required for a good one-dimensional free energy landscape from windowed simulations, for a $D$-dimensional
    problem, we would need to run a reference calculation that involve only $DN_{1D}$ windows, whereas one $D$-dimensional reference free energy
    landscape computed using windowed simulation would require $\mathcal{O}(N_{1D}^D)$ windows. While this may not be a big problem for classical
    MD simulations, the scaling issue can be a big consideration once expensive simulation protocols such as AIMD are involved.

    \subsection{Bounding 1D Umbrella Sampling Errors with EMUS}

    Although umbrella sampling (US) simulations have existed for a long time since the original proposal by Torrie and Valleau, most of the work
    of free energy computations using US rarely published the errors of their surfaces. [CITE] In order to get the free energy out of the 
    windowed simulation, the most commonly used method among researchers is the Weighted-Histogram Analysis Method (WHAM). 
    \cite{P-JCompChem-1992-v13-Kumar} When the error of the US free energy landscape constructed from WHAM is needed, a method called
    Monte Carlo bootstraping analysis is needed, which involves the resampling from generated fake data. [CITE GROSSFIELD WEB] This causes a
    very long time to sample when the resampling size is large. 

    There have also been recent developments in the US simulation as well, and the recent work by Thiede et al. proposed a reformulation of
    the free energy computation from US data into an eigenproblem, which they called the Eigenvector Method for Umbrella Sampling (EMUS).
    \cite{P-JChemPhys-2016-v146-Thiede} They also claimed that EMUS also allowed a computation of the asymptotic variance of the free energy
    landscape without the need to resample any fake data.

    Key to this theory is the expression of the $i$-th normalization constant, $z_i$, as a vector. The definition of $z_i$ is the following,

    \begin{equation}
        z_i = \frac{\int V^b_i(\mathbf{x})\pi(\mathbf{x})d\mathbf{x}}{\sum_{i=1}^{N_{window}}\int V^b_i(\mathbf{x})
            \pi(\mathbf{x})d\mathbf{x}}
    \end{equation}

    \noindent where $V^b_i(\mathbf{x})$ is the $i$-th biased potential in each window, and $\pi(\mathbf{x})$ is the biased probability density
    of finding the configuration $\mathbf{x}$. The free energy in each window is, thus, related to $_i$ as,

    \begin{equation}
        A = -\beta^{-1}\ln z_i
    \end{equation}

    It then can be show that $z_i$ can also be written as a left eigenvector $\mathbf{z}$ of the operator $\mathbf{F}$, where

    \begin{equation}\begin{aligned}
        z_j &= \sum_{i=1}^{N_{window}} z_i F_{ij} \\
        F_{ij} &= \left<\frac{V_j^b}{\sum_{k=1}^{N_{window}}V_j^b}\right>_i
        \label{eq:emus-z}
    \end{aligned}\end{equation}

    Equation \ref{eq:emus-z} indicates that $\mathbf{z}$ can be solved as an eigenvalue problem, and the average free energy for each window
    can then be computed using the knowledge of $\mathbf{z}$. Not only that, the information of $\mathbf{z}$ and $\mathbf{F}$ can also be used
    to estimate the asymptotic variance of EMUS as well according to the central limit theorem. Therefore, for any US simulations, one can
    computed the free energy along with its associated asymptotic variance using EMUS, which perfectly serves as good references for a 
    $D$-dimensional free energy landscape computed using GPR.
\end{spacing}
