\begin{spacing}{2.0}
    \subsection{Computation of the Unbiased Instantaneous Forces (UIFs) from a Biased WT-MTD Simulation}

    We have mentioned earlier in section \ref{sec:free-energy-intro} that a normal, unbiased MD simulation does not sample the rare 
    event regions of the free energy landscape very well due to low probability. Therefore, we would usually get a good information
    in the free energy basins, but the free energy of the rare event regions obtained from unbiased MD simulations tend to be
    noisy due to poor sampling. In order to obtain enough information in the rare event regions without performing windowed simulations,
    guided sampling in the collective variable spaces can be done efficiently with WT-MTD simulations, where an additive
    biasing potential $V_{WT-MTD}(\xi_1,\xi_2,\ldots,\xi_D,t)$ is added to the Hamiltonian over time.

    For any variants of metadynamics, the forms of the potentials similar to equations \ref{eq:metadynamics-potential} and 
    \ref{eq:wt-mtd-potential} are added to the normal potential that is the part of the original Hamiltonian of the system. Therefore, the 
    simulation with these added potential are biased, and the biased total potential for a WT-MTD simulation would have the following form,

    \begin{equation}
        V_{\mathrm{biased}}(\xi_1,\xi_2,\ldots,\xi_D,t) = V(\xi_1,\xi_2,\ldots,\xi_D) + V_{WT-MTD}(\xi_1,\xi_2,\ldots,\xi_D,t)
        \label{eq:biased-potential}
    \end{equation}

    \noindent where $V(\xi_1,\xi_2,\ldots,\xi_D)$ is the potential that is a part of the original Hamiltonian projected onto the collective
    variable space, and is invariant over time. $V_{WT-MTD}$ is defined in equation \ref{eq:wt-mtd-potential}. 

    Equation \ref{eq:uif-definition} outlines the expression of the UIF, which is the negative of the gradient of the original potential
    $V(\xi_1,\xi_2,\ldots,\xi_D)$. By taking the gradients of equation \ref{eq:biased-potential}, we can immediately see that

    \begin{equation}
        -\nabla_{\xi_j}V_{\mathrm{biased}} = -\nabla_{\xi_j}V - \nabla_{\xi_j}V_{WT-MTD}
        \label{eq:potential-gradients}
    \end{equation}

    Let $\phi_{\mathrm{biased}}(\xi_j) = -\nabla_{\xi_j}V_{\mathrm{biased}}$ and $\phi_{WT-MTD}(\xi_j) = -\nabla_{\xi_j}V_{WT-MTD}$, then 
    equation \ref{eq:potential-gradients} becomes,

    \begin{equation}\begin{aligned}
        \phi_{\mathrm{biased}}(\xi_j) &= \phi(\xi_j) + \phi_{WT-MTD}(\xi_j) \\
        \phi(\xi_j) &= \phi_{\mathrm{biased}}(\xi_j) - \phi_{WT-MTD}(\xi_j)
    \end{aligned}\end{equation}

    Since the computation of the free energy requires $\phi(\xi_j)$ as a free energy estimator for the $\xi_j$ dimension, 
    we need to compute $\phi_{\mathrm{biased}}(\xi_j)$ and $\phi_{WT-MTD}(\xi_j)$ from a biased WT-MTD simulation. 
    $\phi_{\mathrm{biased}}(\xi_j)$ is computed from $V_{\mathrm{biased}}$ by a transformation of $V_{\mathrm{biased}}$
    from a Cartesian coordinate system usually common in frequently used MD software packages into the collective variable space using
    the following equation, \cite{B-FEComput}

    \begin{equation}
        \mathbf{f}_{\mathrm{biased}} = -(\mathbf{G}_W^{-1}\mathbf{W})\cdot\nabla_{\mathbf{X}}V(\mathbf{X}) + 
            \beta^{-1}\nabla_{\mathbf{X}}\cdot(\mathbf{G}^{-1}_W\mathbf{W})
        \label{eq:uif-second-derivative}
    \end{equation}

    \noindent where $\mathbf{f}_{\mathrm{biased}} = \left[\phi_{\mathrm{biased}}(\xi_1)\,\phi_{\mathrm{biased}}(\xi_2)\,
    \ldots\,\phi_{\mathrm{biased}}(\xi_D)\right]^{\top}$ is a $D\times 1$ vector of the biased instantaneous forces (BIFs), 
    $\mathbf{G}_W^{-1} = \mathbf{W}\nabla_{\mathbf{X}}\xi$ is the generalized Gram matrix, $\nabla_{\mathbf{X}}\xi$ is the Jacobian of the
    collective variables, which is a $3N_{atoms} \times D$ matrix, and $\mathbf{W} = \nabla_{\mathbf{X}}\xi^{\top}\mu^{-1}$, where 
    $\mu^{-1} = \delta_{ij}m_i^{-1}$ is a $3N_{atoms} \times 3N_{atoms}$ matrix whose diagonal elements represent masses of each atom
    in the system. However, one could immediately see that the second term of equation \ref{eq:uif-second-derivative} requires a computation of a 
    Hessian matrix, which can be expensive in case a collective variable involves many atoms. Nevertheless, Darve et al. has pointed out the 
    relationship between equation \ref{eq:uif-second-derivative} and the average of the potential’s gradient. 
    \cite{P-JChemPhys-2008-v128-Darve} Therefore, equation \ref{eq:uif-second-derivative} can be simplified as follow,

    \begin{equation}
        \mathbf{f}_{\mathrm{biased}} = \frac{d}{dt}(\mathbf{W}\nabla_{\mathbf{X}}\xi)^{-1}\frac{d\xi}{dt}
        \label{eq:uif-first-derivative}
    \end{equation}

    Equation \ref{eq:uif-first-derivative} can be computed numerically provided that the biased trajectory is evenly spaced by a small enough 
    timestep $\Delta t$ to minimize the numerical error. Therefore, a vector $\mathbf{f} = \left[\phi(\xi_1)\,\phi(\xi_2)
    \,\ldots\,\phi(\xi_D)\right]^{\top}$ of the UIFs can be computed by subtracting $\mathbf{f}_{\mathrm{biased}}$ with
    $\mathbf{f}_{WT-MTD} = \left[\phi_{WT-MTD}(\xi_1)\,\phi_{WT-MTD}(\xi_2)\,\ldots\,\phi_{WT-MTD}(\xi_D)\right]^{\top}$, which could be
    easily computed by simply taking the gradient of $V_{WT-MTD}$, already expressed in terms of the collective variables of interests.

    \begin{equation}
        \mathbf{f}_{WT-MTD} = -\nabla_{\xi}V_{WT-MTD}
    \end{equation}

    With $\mathbf{f}$ determined, a training set of the collective variable points and the unbiased instantaneous forces can be built for
    GPR reconstruction of a $D$-dimensional free energy landscape. 
\end{spacing}
