\begin{spacing}{2.0}
    \section{GPR and Free Energy Computation}

    \subsection{Fast Exploration of the Collective Variable Space}

    Since the inherent statistical noises of the free energy estimators are already included in the formulation of GPR, we do not need to perform 
    expensive multidimensional windowed simulations to minimize the statistical noise as in any regular US or TI variants. As mentioned earlier, 
    the free energy estimator can either be local probability density $p_i(\xi_1,\xi_2,\ldots,\xi_D)$ or the mean force 
    $f_i(\xi_j) = -\left<\nabla_{\xi_j}V\right>$. However, GPR already took care of the noise in its algorithm; therefore, in principle, the mean 
    force does not need to be averaged despite an individual expression of the force along a particular collective variable would imply a relatively 
    large amount of noise, and the unbiased instantaneous forces (UIFs), $\phi$ can thus be used as a noisy free energy estimator for GPR,

    \begin{equation}
        \phi(\xi_j) = -\nabla_{\xi_j}V(\xi_1,\xi_2,\ldots,\xi_D)
        \label{eq:uif-definition}
    \end{equation}

    Removing the need to perform windowed simulations is the main source of the improved efficiency for using GPR to reconstruct free energy
    landscapes. Consequently, in order to get the free energy landscape that covers a specific area in the collective variable space, it is
    logical to devise a strategy that would allow us to achieve fastest sampling of the configuration space to save the simulation efforts.
    Cuendet and Tuckerman \cite{P-JCTC-2014-v10-Cuendet} suggested that MTD, TAMD, or adiabatic free energy dynamics can be used for guiding
    the exploration of the configuration space, and Mones et al. \cite{P-JCTC-2016-v12-Mones} suggested that using well-tempered metadynamics
    (WT-MTD), a variant of metadynamics, resulted in the fastest exploration of the regions of interest in their two and four-dimensional
    free energy landscapes of alanine peptides. 
\end{spacing}
