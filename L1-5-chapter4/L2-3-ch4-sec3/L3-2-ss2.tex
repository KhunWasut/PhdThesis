\begin{spacing}{2.0}
    \subsection{Well-Tempered Metadynamics as an Exploration Tool}

    To understand why WT-MTD allows a speedy exploration of the configuration space, it is important to first understand the underlying 
    theoretical aspects behind regular metadynamics. Originally proposed by Laio and Parrinello in 2002, regular metadynamics adds repulsive
    Gaussians $V_{MTD}$ of the following form over a long period of simulation, \cite{P-PNAS-2002-v99-Laio}

    \begin{equation}
        V_{MTD}(\xi_1,\xi_2,\ldots,\xi_D,t) = \sum_{t=0,\,\Delta t,\,2\Delta t,\,\ldots,\,N_G\Delta t} 
            h\exp\left(-\frac{1}{2}\sum_{i=1}^D\frac{\left(\xi_i-\xi_i(t)\right)^2}{\sigma_i^2}\right)
        \label{eq:metadynamics-potential}
    \end{equation}

    \noindent where $h$ represents the height of each Gaussian, which is constant over time, $N_G$ is the number of deposited Gaussians,
    and $\sigma_i^2$ is the width of the Gaussian in the $i$-th dimension. If more Gaussians are deposited in a metadynamics simulation, the 
    barrier would become more accessible due to less difference in the potential energy. According to equation \ref{eq:free-energy-prob-defn}, 
    this would mean that the probability of accessing the barrier would exponentially increase, and if any metadynamics simulations are performed 
    for a long period of time, the deposited Gaussians would converge to the negative value of the free energy.

    \begin{equation}
        V_{MTD}(\xi_1,\xi_2,\ldots,\xi_D, t\to\infty) = -A(\xi_1,\xi_2,\ldots,\xi_D) + C
    \end{equation}

    The rate of the convergence of metadynamics depend on our choice of the parameters $h$ and $\sigma_i^2$. While $\sigma_i^2$ governs 
    the rate of exploration by dictating how wide of the area in its collective space domain that the Gaussians shall cover, $h$
    governs how high the Gaussians would be. If $h$ is not chosen carefully, then it is possible that the converged $V_{MTD}$ may become
    noisy and not smooth. In order to ameliorate this issue, well-tempered metadynamics (WT-MTD) was introduced by Barducci et al.
    to ensure that over time, $h$ would steadily be decreasing until it asymtotically converges to zero. The bias potential form
    of WT-MTD takes a similar form with equation \ref{eq:metadynamics-potential}, with a slight difference.

    \begin{equation}
        V_{WT-MTD}(\xi_1,\xi_2,\ldots,\xi_D,t) = \sum_{t=0,\,\Delta t,\,2\Delta t,\,\ldots,\,N_G\Delta t} 
            h(t)\exp\left(-\frac{1}{2}\sum_{i=1}^D\frac{\left(\xi_i-\xi_i(t)\right)^2}{\sigma_i^2}\right)
        \label{eq:wt-mtd-potential}
    \end{equation}

    Instead of being a constant, the Gaussian height in WT-MTD simulations, $h(t)$, decays over time according to the following equation,

    \begin{equation}
        h(t) \propto \exp\left(-\frac{V_{WT-MTD}(\xi_1,\xi_2,\ldots,\xi_D,t)}{k_B\Delta T}\right)
        \label{eq:wt-mtd-decay}
    \end{equation}

    The exponential term in equation \ref{eq:wt-mtd-decay} tells us that as the Gaussians are deposited, $h(t)$ would be modified by
    the exponential factor of the negative value of all the deposited Gaussians at that time. Hence, as $t\to\infty$, $h(t)$ would
    asymptotically converge to zero, which indicates the point where the deposited Gaussians are appropriately smoothened, eliminating
    the noises that may present from using the regular variant of metadynamics. The rate at which $h(t)$ decays over time is governed
    by an additional parameter, $\Delta T$, which is called the well-tempered temperature. In WT-MTD simulations, $\Delta T$ can be
    thought of as a high temperature value that adds up to the regular, thermostatted simulation temperature that modifies the 
    probability of finding a particular configuration in equation \ref{eq:free-energy-prob-defn} to the following,

    \begin{equation}
        p(1\to *) \propto e^{-\beta^*\Delta A_{t\to *}}
    \end{equation}

    \noindent where $\beta^* = \left[k_B(T + \Delta T)\right]^{-1}$. This means when $\Delta T$ is high, the probability of sampling
    the rare event is significantly greater, while when $\Delta T \to 0$, the WT-MTD simulation would converge to a normal MD simulation
    in a canonical ensemble. A value that is commonly used in literature to indicate the strength of the WT-MTD bias is called a 
    \textsl{bias factor}, $\gamma$, which is defined as follow, \cite{P-JCTC-2016-v12-Sun}

    \begin{equation}
        \gamma = \frac{T + \Delta T}{T}
    \end{equation}

    However, the drawback of WT-MTD is that the Gaussians do not converge to the free energy as in regular metadynamics. Rather,
    it converges to a factor of the free energy, and the factor is determined by the choice of $\Delta T$, or $\gamma$.

    \begin{equation}\begin{aligned}
        V_{WT-MTD}(\xi_1,\xi_2,\ldots,\xi_D,t\to\infty) &= -\frac{\Delta T}{T + \Delta T}A(\xi_1,\xi_2,\ldots,\xi_D) + C \\
            &= -\frac{\gamma - 1}{\gamma}A(\xi_1,\xi_2,\ldots,\xi_D) + C
    \end{aligned}\end{equation}

    Nevertheless, WT-MTD still retains the same advantage as regular metadynamics that the Gaussian deposition is implicitly guided by
    the original probability of finding a configuration (equation \ref{eq:free-energy-prob-defn}). Thus, the Gaussian deposition would
    still be aimed more toward any regions with deep free energy minima, and will still leveling out those deep wells so that the
    simulation would move across the free energy barriers more frequently. The exploration is further aided by our choice of $\gamma$.
    For the purpose of the free energy computations, the accepted range of $\gamma$ for WT-MTD simulation is between 10 to 15.
    Using a significantly higher $\gamma$ than this range would still result in a noisy reconstruction of the free energy landscape,
    an issue that usually presents with the regular variant of metadynamics. However, using a very high value of $\gamma$ has
    a benefit for exploration of the configuration space of any systems with deep minima and high free energy barriers, because
    the exploration can be done far more quickly while the Gaussian heights would also be partially decayed. Mones et al. reported
    their exploration of the free energy landscapes of alanine peptides with a value of $\gamma = 33.3$ 
    ($T = 300\,\mathrm{K}$ and $\Delta T = 10,000\,\mathrm{K}$), 
    which is unsuitable for recovering the free energy, but hugely aids the exploration efforts. \cite{P-JCTC-2016-v12-Mones}
\end{spacing}
