% ABSTRACT OF THE DISSERTATION
\newpage

\begingroup
\vspace*{1.5in}
\begin{center}
    {ABSTRACT OF THE DISSERTATION}
\end{center}
\addcontentsline{toc}{chapter}{Abstract of the Dissertation}
\endgroup

\vspace{1.2cm}

\begin{center}
    \begin{spacing}{1.0}
    {Theoretical investigations of ionic association reactions in aqueous solutions: \\
    Using Markov State Model to determine a proper reaction coordinate and \\
    machine-learned two-dimensional free energy landscapes with \\
    Gaussian Process Regression}
    \end{spacing}

    \vspace{0.6cm}

    by

    \vspace{0.6cm}

    \begin{spacing}{1.0}
        Wasut Pornpatcharapong

        Doctor of Philosophy in Chemistry

        University of California, San Diego, 2018
    \end{spacing}

    \vspace{0.6cm}

    Professor John Weare, Chair
\end{center}

\vspace{0.6cm}

% DUMMY TEXT TO MODEL THE ABSTRACT AT SOME LENGTH
\begin{spacing}{2.0}
    Understanding a proper reaction coordinate and the free energy profile of any chemical reaction provide valuable information in elucidating 
    the kinetics and thermodynamics properties, as well as the underlying reaction mechanisms. Nevertheless, identifying a proper reaction 
    coordinate for a specific chemical reaction or computing the free energy landscape are difficult. Hence, any methods that could systematically 
    provide insights into both issues would play important roles in studies of any kind of chemical reactions.

    On the reaction coordinate’s front, Markov State Model (MSM) is a tool that can be used to identify the reaction coordinate of the slowest 
    motion in a simulation. Instead of the deterministic view of the MD simulations, MSM takes a probabilistic view that a configuration has a 
    probability to evolve to another configuration after time t defined by a transfer operator, which allows us to identify each motion in the 
    system in terms of the information encapsulated in each of the operator’s corresponding eigenfunctions and eigenvalues. The eigenfunctions 
    of the transfer operator can then be projected onto the collective variable space, and minimal representation of each eigenfunction in the 
    collective variable space could be obtained.

    On the free energy front, efficient multidimensional free energy landscape can be reconstructed smoothly from noisy free energy estimators 
    through Gaussian Process Regression (GPR). In this dissertation, we proposed a rigorous GPR workflow that also ensures the consistency through 
    projection of the multidimensional landscape into each individual one-dimensional surface with errors bounded by Eigenvector Method for 
    Umbrella Sampling (EMUS).

    This dissertation employed both MSM and GPR to study the dynamics of cation---anion association in aqueous solutions using LiCl and NaCl as 
    a model. With MSM, we have completely identified all significant motions in the association process and in the bulk, and we also identify 
    important contributions to the slowest process of the dynamics. With GPR, we have achieved a smooth reconstruction of a free energy landscape 
    of both systems using 2 collective variables and large efficiency gain relative to traditional two-dimensional windowed simulations. 
\end{spacing}
