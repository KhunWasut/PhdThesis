\begin{spacing}{2.0}
    \chapter{Ionic Association of NaCl in Aqueous Solution}

    \section{Introduction}

    The Markovian interpretation of MD simulations opened a wide avenue for calculating many interesting properties from a rigorous mathematical 
    formalisms introduced in Chapter \ref{chap:chapter-msm}. Nevertheless, there are many components in this formalism, such as Time-lagged 
    Independent Component Analysis (TICA), Matching Pursuit (MP), Markov State Model (MSM), or the recently published concepts of commute maps and 
    commute distances. [CITE WORKS] Therefore, often times, it is difficult for researchers to grasp the whole concept of the Markovianity of MD 
    simulations, preventing them to fully utilize the potentials. Recently, there have been significant breakthroughs in the applications of MSM 
    and TICA in numerous simulations for biological systems, where the metastable states of the biomolecular configurations and the mechanisms and 
    the probability of transition between the metastable states can be computed. [CITE] However, in our opinion, the entire package has rarely been 
    used outside the biomolecuar simulation community. Hence, we believe that the interpretations of the Markovianity of chemical reactions can be 
    applied to solve numerous problems in chemical dynamics, especially for problems that are easier to solve and a large enough amount of data 
    can be obtained to assure that a simulation was performed to conform with an equilibrium under the canonical ensemble.

    Therefore, we chose to apply the formulisms of Chapter \ref{chap:chapter-msm} to study the ionic association of NaCl in aqueous solution. 
    Despite being a relatively simple problem, the mechanistic point of view of ionic association in aqueous solution remains relatively little 
    understood. The work of Mullen et al. has opened new insights to this type of problem through the introduction of the $n_B$ collective variable 
    describing the number of water molecules simultaneously associating with both the cation and the anion. \cite{P-JCTC-2014-v10-Mullen} This 
    finding challenged the prevalent views of this process from the ligand exchange perspective, where the association of the ions is driven by 
    the loss of water molecule from the cation's first solvation shell in order to make room for the CIP configuration. The identification of $n_B$
    as an important collective variable; thus, is an important step towards better understandings of this problem. Nevertheless, McGibbon et al. 
    proposed that a true reaction coordinate representing the slowest motion of the dynamics must rigorously follow the following three 
    conditions, \cite{P-JChemPhys-2017-v146-McGibbon}

    \begin{enumerate}
        \item{It has to be at a reduced dimension from $\Omega \to \mathbb{R}$}
        \item{It needs to be uniquely determined by the dynamics rather than being conditions enforced \textsl{a priori}}
        \item{It needs to be a maximally predictive projection}
    \end{enumerate}

    In order to satisfy the three conditions above without any preconditioning of the committor surfaces a priori, we opted for the Markovian 
    formulisms proposed in the previous chapter, where the slowest motion of the dynamics can be readily determined from the eigenfunction of the 
    transfer operator with a corresponding highest eigenvalue that is not 1. From this formulism, the variational principle projection of the 
    slowest eigenfunction can be done onto the basis set of collective variables through TICA, where the linear combination of the collective 
    variables constitute a TICA eigenfunction, or a reaction coordinate. The TICA reaction coordinates can also be further reduced with MP to form 
    a more physically interpretable version of the reaction coordinates focusing only on a few relevant collective variables. Having the 
    information of $m$ slowest reaction coordinates from TICA that covers the majority of the cumulative kinetic variance, a good MSM for this 
    system can then be built based on the coordinates that fully cover the extent of the simulation, from where the transition matrix between the 
    coarsed metastable states can be computed either through the Hidden Markov Model (HMM), or Perron-Cluster Cluster Analysis (PCCA), allowing us 
    to calculate the transition rate between relevant metastable states and the commute map of this system. Thus, here we present the Markovian 
    interpretation of the ionic association of NaCl in an aqueous solution.
\end{spacing}
