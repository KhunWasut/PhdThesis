\begin{spacing}{2.0}
    \section{Summary}

    This chapter presents the Markovian interpretation of the dynamics of the ionic association of NaCl in aqueous solutions, where slowest reaction 
    coordinates for this process is determined using TICA. We found that there are 6 coordinates that sums up the cumulative variance for up to 0.997,
    while the dynamics are mostly dominated by two slowest motions with similar relaxation timescales.
    These 6 reaction coordinates are then subsequently used for building the MSM of this process using $k$-means discretization.
    Further coarse-graining of the MSM transition probability matrix was done through PCCA, and the kinetic profile of this reaction was then reconstructed
    with a Hidden Markov Model.

    The interpretation of a 6-state HMM allows us to deduce the arrangement of the CIP, SSIP, and the bulk states in the collective variable space,
    as well as the probability of the dynamics staying in each specific state. We found that for NaCl, the dynamics greatly prefers the bulk, and the 
    contacted ion pair is very rarely formed. The 20-state HMM allows a better assignment of the metastable states, and with the projected free energy
    landscapes in appropriate coordinates, more information of the dynamics and the mechanistic point of view for this reaction can be determined. 
    The projection of the 20-state HMM onto the free energy landscape in two of the slowest reaction coordinates found from TICA verifies the existence
    of the CIP, SSIP, and the bulk states, and further projection onto the appropriate collective variables allow us to interpret the physical and 
    mechanistic meaning of the two slowest motions in this system, where the reaction coordinate $\tilde{\psi}_2$ drives the ionic association through
    the local effect of bridge formation of water molecules that simultaneously coordinate with both the ions, while the reaction coordinate $\tilde{\psi}_3$
    drives the ionic association through a more global effect by the expulsion of solvent molecules away from the region between the two ions as the
    two ions come into a closer contact.

    Despite the rigor of the theory, the main drawback here is the computation of the free energy from the histograms of the molecular dynamics trajectory.
    The areas around the barrier can never be thoroughly sampled without a guided scheme, and the information around the transition state can deviate,
    causing the incorrect inference of the mechanism and the rate. The next two chapters will present a scheme that can be used to calculate multidimensional
    free energy landscape with less computational resources, while producing quantitatively plausible results through the application of Gaussian Process
    Regression (GPR).
\end{spacing}
