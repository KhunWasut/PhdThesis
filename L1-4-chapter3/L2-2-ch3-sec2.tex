\section{Simulation Details}
\begin{spacing}{2.0}
    \subsection{General Settings}

    The system, \ce{NaCl + 495H2O}, contains one Na cation, one Cl anion, and 495 water molecules, was prepared by placing the ions randomly in a 
    box of 30.0 \r{A}, and the water molecules were then placed in the same box using PACKMOL. \cite{P-JCompChem-2009-v30-Martinez} In order to 
    get an equilibrium box size at 1.0 atm, the simulation in an isothermic-isobaric ensemble was performed for 960 ps using Nos\'{e}-Hoover 
    Langevin piston \cite{P-JChemPhys-1994-v101-Martyna, P-JChemPhys-1995-v103-Feller} Once the equilibrium box size is obtained, the production 
    simulation was performed under the canonical ensemble using Langevin dynamics with a damping constant of 5.0 $\mathrm{ps}^{-1}$ with NAMD 
    \cite{P-JCompChem-2005-v26-Phillips} with the simulation timestep of 2.0 fs. Water molecules in this simulation are treated as rigid under a 
    TIP3P model \cite{P-JChemPhys-1983-v79-Jorgensen}, and the force field parameters for all atoms are derived from the work by Joung and Cheatam. 
    \cite{P-JPhysChemB-2008-v112-Joung} The electrostatic interactions were modeled by Particle Mesh Ewald \cite{P-JChemPhys-1993-v98-Darden} 
    algorithm. The temperature of the simulation was controlled at 300 K, and periodic boundary conditions were applied throughout the simulation.
    The trajectory was saved every 10 steps. The total simulation time for this system is 220 ns.

    The analysis part started from reading the entire trajectory with MDTraj \cite{P-BiophysJ-2015-v109-McGibbon} using a 10-frame stride; therefore,
    each of the snapshot input is 100 steps, or 0.2 picoseconds apart from one another. The analysis started with finding the slowest 
    variational eigenfunctions $\tilde{\psi}_i$ with TICA at different lag times $\tau$, from where we picked a value of $\tau$ such that 
    all the $m-1$ eigenfunctions consist of different motions that cover largest extent of the kinetic map possible. In order to form a Markov
    State Model (MSM) for this reaction, $k$-means clustering [CITE] with $k$-means++ was used to discretize the input data into the space of 
    $m-1$ $\tilde{\psi}_i$ obtained earlier from TICA, from which a coarse kinetic model, transition matrix, and commute maps can be obtained 
    from Perron-Cluster Clustering Approach (PCCA+) [CITE] or Hidden Markov Model (HMM) [CITE]. All the analysis from TICA to coarse-grained
    MSM was performed with a software package PyEMMA. \cite{P-JCTC-2015-v11-Scherer}

    \subsection{Collective Variables}

    In order to get a best reaction coordinate, a good set of collective variables are needed so that there are more basis functions available for
    TICA. For this work, we chose two different sets of the collective variables for a comparison purpose. One set (CV Set 1) is a set of 13 
    collective variables modeled after the work by Mullen et al., and another set (CV Set 2) is a set of 32 intuition-based variables that are 
    thought to describe both the ionic and the solvent coordinates. Tables \ref{tab:cv-set-1} and YY show brief descriptions of the chosen 
    collective variables for this reaction, as well as their corresponding index numbers. Based on our choice for the collective variables, they 
    can be further subdivided into the following classes,

    \begin{table}[h]
        \centering
        \caption{A list of collective variables used in CV Set 1 inspired by the previous work in Mullen et al.}
        \label{tab:cv-set-1}
        \begin{tabular}{|c|c|c|c|}
            \hline
            \textbf{Index} & \textbf{Feature} & \textbf{Feature Class} & \textbf{Remarks} \\ \hline
            1 & $n_+^{(1)}$ & Ion Coordination & \\
            2 & $n_-^{(1)}$ & Ion Coordination & \\
            3 & $n_B$ & Water Between Ions & \\
            4 & $n_+^{(2)}$ & Ion Coordination & Including first shell \\ 
            5 & $n_-^{(2)}$ & Ion Coordination & Including first shell \\
            6 & $n_+^{(2)}$ & Ion Coordination & Excluding first shell \\ 
            7 & $n_-^{(2)}$ & Ion Coordination & Excluding first shell \\
            8 & $r_{+-}$ & Interionic Separation & \\
            9 & $\rho_{ii}$ & Water Density & $\sigma = 3.57$ \r{A} \\
            10 & $\rho_{ii}$ & Water Density & $\sigma = r_{+-}/4$ \\
            11 & $\rho_{ii}$ & Water Density & $\sigma = r_{+-}/3$ \\
            12 & $\rho_{ii}$ & Water Density & $\sigma = r_{+-}/2$ \\
            13 & $\rho_{ii}$ & Water Density & $\sigma = r_{+-}$ \\ \hline
        \end{tabular}
    \end{table}

    \textsl{The Interionic Separation Class} - This class of the collective variable is based on the interionic separation between the cation and 
    the anion, which is defined to be the Euclidean distance between the two ions in the simulation,

    \begin{equation}
        r_{+-} = \left\lVert \mathbf{r}_{\mathrm{Na}} - \mathbf{r}_{\mathrm{Cl}} \right\rVert
        \label{eq:r-definition}
    \end{equation}

    The collective variables belonging to this class are $r_{+-}$ itself, as well as its derivatives, such as $\frac{1}{r_{+-}}$ or 
    $\frac{1}{r_{+-}^2}$, which are defined in the case that the derivatives can describe the dynamics better than the original variable.

    \textsl{The Ionic Coordination Class} - This class of the collective variables aim to describe the behavior relating to the first and the second 
    solvation shells of the cation and the anion to monitor the possibility of the ligand exchange type of reactions involved in the ion pairing. 
    For CV Set 1, the definition of the ionic coordination class collective variables are taken straight from the work of Mullen et al., with slight 
    modification of parameters to suit our simulation better,

    \begin{equation}\begin{aligned}
        n_+^{(i)} &= \sum_{j=1}^{N_{wat}}\frac{1-\tanh\left[\alpha\left(\left\lVert\mathbf{r}_{\mathrm{Na-O}_j}\right\rVert - b_{\mathrm{Na}}^{(i)}\right)\right]}{2} \\
        n_-^{(i)} &= \sum_{j=1}^{N_{wat}}\frac{1-\tanh\left[\alpha\left(\left\lVert\mathbf{r}_{\mathrm{Cl-H}_j}\right\rVert - b_{\mathrm{Cl}}^{(i)}\right)\right]}{2}
    \end{aligned}\end{equation}

    \noindent where $n^{(i)}$ represents the $i$-th solvation shell of a corresponding ion, $\mathbf{r}_{\mathrm{Na-O}_j}$ is simply 
    $\mathbf{r}_{\mathrm{Na}} - \mathbf{r}_{\mathrm{O}}$, $\mathbf{r}_{\mathrm{Cl-H}_j} = \mathbf{r}_{\mathrm{Cl}} - \mathbf{r}_{\mathrm{H}}$ where 
    $\mathbf{r}_{\mathrm{H}}$ is the vector that points to the nearest hydrogen atom of each water molecule to \ce{Cl^-}, and $b^{(i)}$ is the
    $i$-th minimum of the radial distribution function between the ions and the water molecules. For this simulation, $b_{\mathrm{Na}}^{(1)} = 3.18$ \r{A},
    $b_{\mathrm{Na}}^{(2)} = 5.88$ \r{A}, $b_{\mathrm{Cl}}^{(1)} = 3.98$ \r{A}, and $b_{\mathrm{Cl}}^{(2)} = 6.28$ \r{A}. The constant $\alpha$ also
    varies with the ions and the solvation shell, where $\alpha_{\mathrm{Na}}^{(1)} = 3$, $\alpha_{\mathrm{Na}}^{(2)} = 12$, $\alpha_{\mathrm{Cl}}^{(1)} = 7$,
    and $\alpha_{\mathrm{Cl}}^{(2)} = 15$. %For CV Set 2, there are also another alternative which has the following form,

    %\begin{equation}\begin{aligned}
    %    n_+^{(i)} &= \sum_{j=1}^{N_{wat}} \frac{1-\left(\frac{\left\lVert\mathbf{r}_{\mathrm{Na-O}_j}\right\rVert}{b_{\mathrm{Na}}^{(i)}}\right)^n}
    %        {1-\left(\frac{\left\lVert\mathbf{r}_{\mathrm{Na-O}_j}\right\rVert}{b_{\mathrm{Na}}^{(i)}}\right)^m} \\
    %    n_-^{(i)} &= \sum_{j=1}^{N_{wat}} \frac{1-\left(\frac{\left\lVert\mathbf{r}_{\mathrm{Cl-H}_j}\right\rVert}{b_{\mathrm{Cl}}^{(i)}}\right)^n}
    %        {1-\left(\frac{\left\lVert\mathbf{r}_{\mathrm{Cl-H}_j}\right\rVert}{b_{\mathrm{Cl}}^{(i)}}\right)^m} \\
    %    \label{eq:CN-power}
    %\end{aligned}\end{equation}

    %\noindent where $n << m$. The derivatives of these variables are also considered to be belong to this class as well. The definition in equation
    %\ref{eq:CN-power} was widely used in numerous literatures. [CITE] \placeholder{We also have other definitions as well.}

    \textsl{The Water Between Ions Class} - This class of the collective variables was first conceived in Mullen et al. \cite{P-JCTC-2014-v10-Mullen}, which
    aims to describe the simultaneous association of a water molecule with respect to either of the ion. Therefore, for higher $r_{+-}$, there should be
    no water molecules associating with both of the ions. However, for small $r_{+-}$, more water molecules can associate with the ions, bridging the
    ions together. The definition of $n_B$, the number of water molecules associating with both of the ions, is defined as follow,

    \begin{equation}
        n_B = \sum_{j=1}^{N_{wat}}\max\left(\frac{1-\tanh\left[\alpha\left(\left\lVert \mathbf{r}_{\mathrm{Na-O}_j} \right\rVert - b_{\mathrm{Na}}^{(1)}\right)\right]}{2}, 
            \frac{1-\tanh\left[\alpha\left(\left\lVert \mathbf{r}_{\mathrm{Cl-H}_j} \right\rVert - b_{\mathrm{Cl}}^{(1)}\right)\right]}{2}\right)
    \end{equation}

    \noindent where all the parameters, $\alpha$, $b_{\mathrm{Na}}^{(1)}$, $b_{\mathrm{Cl}}^{(1)}$ are the same as the ones defined for the ionic coordination
    class. This variable, $n_B$, is used in CV Set 1. %However, for CV Set 2, we use another variable per suggestions from Professor Cecilia Clementi
    %that indicates the number of water molecules between the two ions

    \textsl{The Water Density Class} - This class of collective variables describes the density of water molecules around the midpoint between
    the two ions. In Mullen et al., this variable is defined as follow,

    \begin{equation}
        \rho_{ii} = \frac{1}{(2\pi\sigma^2)^{3/2}}\sum_{j=1}^{N_{wat}}\exp\left[-\frac{\left\lVert \mathbf{r}_{\mathrm{O}_j} - \mathbf{r}_{mid} \right\rVert^2}
            {2\sigma^2}\right]
    \end{equation}

    \noindent where $\mathbf{r}_{mid}$ is the vector that points to the midpoint between the two ions. This variable has a unit of $\mathrm{length}^{-3}$, and
    there are 5 variables in CV Set 1 belonging to this class, each of which has a different value of $\sigma$, which indicates how rapidly the density would
    vary around the midpoint between the ions. The smaller the value of $\sigma$, the density varies more abruptly. Another way to look at this variable is
    that it represents the number of water molecules inside the volume $V_{ii} = \left(2\pi\sigma^2\right)^{3/2}$. In this work, we used the values of $\sigma$
    at 3.54 \r{A}, $r_{+-}/4$, $r_{+-}/3$, $r_{+-}/2$, and $r_{+-}$. 
\end{spacing}
