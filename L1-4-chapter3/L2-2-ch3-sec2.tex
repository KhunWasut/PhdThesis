\section{Simulation Details}
\begin{spacing}{2.0}
    \subsection{General Settings}

    The system, \ce{NaCl + 495H2O}, contains one Na cation, one Cl anion, and 495 water molecules, was prepared by placing the ions randomly in a 
    box of 30.0 \r{A}, and the water molecules were then placed in the same box using PACKMOL. \cite{P-JCompChem-2009-v30-Martinez} In order to 
    get an equilibrium box size at 1.0 atm, the simulation in an isothermic-isobaric ensemble was performed for 960 ps using Nos\'{e}-Hoover 
    Langevin piston \cite{P-JChemPhys-1994-v101-Martyna, P-JChemPhys-1995-v103-Feller} Once the equilibrium box size is obtained, the production 
    simulation was performed under the canonical ensemble using Langevin dynamics with a damping constant of 5.0 $\mathrm{ps}^{-1}$ with NAMD 
    \cite{P-JCompChem-2005-v26-Phillips} with the simulation timestep of 2.0 fs. Water molecules in this simulation are treated as rigid under a 
    TIP3P model \cite{P-JChemPhys-1983-v79-Jorgensen}, and the force field parameters for all atoms are derived from the work by Joung and Cheatam. 
    \cite{P-JPhysChemB-2008-v112-Joung} The electrostatic interactions were modeled by Particle Mesh Ewald \cite{P-JChemPhys-1993-v98-Darden} 
    algorithm. The temperature of the simulation was controlled at 300 K, and periodic boundary conditions were applied throughout the simulation.

    \subsection{Collective Variables}

    In order to get a best reaction coordinate, a good set of collective variables are needed so that there are more basis functions available for
    TICA. For this work, we chose two different sets of the collective variables for a comparison purpose. One set (CV Set 1) is a set of 13 
    collective variables modeled after the work by Mullen et al., and another set (CV Set 2) is a set of 32 intuition-based variables that are 
    thought to describe both the ionic and the solvent coordinates. Tables XX and YY show brief descriptions of the chosen collective variables for 
    this reaction, as well as their corresponding index numbers. Based on our choice for the collective variables, they can be further subdivided 
    into the following classes,

    \textsl{The Interionic Separation Class} - This class of the collective variable is based on the interionic separation between the cation and 
    the anion, which is defined to be the Euclidean distance between the two ions in the simulation,

    \begin{equation}
        r_{+-} = \left\lVert \mathbf{r}_{\mathrm{Na}} - \mathbf{r}_{\mathrm{Cl}} \right\rVert
    \end{equation}

    The collective variables belonging to this class are $r_{+-}$ itself, as well as its derivatives, such as $\frac{1}{r_{+-}}$ or 
    $\frac{1}{r_{+-}^2}$, which are defined in the case that the derivatives can describe the dynamics better than the original variable.

    \textsl{The Ionic Coordination Class} - This class of the collective variables aim to describe the behavior relating to the first and the second 
    solvation shells of the cation and the anion to monitor the possibility of the ligand exchange type of reactions involved in the ion pairing. 
    For CV Set 1, the definition of the ionic coordination class collective variables are taken straight from the work of Mullen et al., with slight 
    modification of parameters to suit our simulation better,

    \begin{equation}\begin{aligned}
        n_+^{(i)} &= \sum_{j=1}^{N_{wat}}\frac{1-\tanh\left[\alpha\left(\left\lVert\mathbf{r}_{\mathrm{Na-O_j}}\right\rVert - b_{\mathrm{Na}}^{(i)}\right)\right]}{2} \\
        n_-^{(i)} &= \sum_{j=1}^{N_{wat}}\frac{1-\tanh\left[\alpha\left(\left\lVert\mathbf{r}_{\mathrm{Cl-H_j}}\right\rVert - b_{\mathrm{Cl}}^{(i)}\right)\right]}{2}
    \end{aligned}\end{equation}

    \noindent where $n^{(i)}$ represents the $i$-th solvation shell of a corresponding ion, $\mathbf{r}_{\mathrm{Na-O}_j}$ is simply 
    $\mathbf{r}_{\mathrm{Na}} - \mathbf{r}_{\mathrm{O}}$, $\mathbf{r}_{\mathrm{Cl-H}_j} = \mathbf{r}_{\mathrm{Cl}} - \mathbf{r}_{\mathrm{H}}$ where 
    $\mathbf{r}_{\mathrm{H}}$ is the vector that points to the nearest hydrogen atom of each water molecule to \ce{Cl^-}, and $b^{(i)}$ is the
    $i$-th minimum of the radial distribution function between the ions and the water molecules. For this simulation, $b_{\mathrm{Na}}^{(1)} = 3.18$ \r{A},
    $b_{\mathrm{Na}}^{(2)} = 5.88$ \r{A}, $b_{\mathrm{Cl}}^{(1)} = 3.98$ \r{A}, and $b_{\mathrm{Cl}}^{(2)} = 6.28$ \r{A}. The constant $\alpha$ also
    varies with the ions and the solvation shell, where $\alpha_{\mathrm{Na}}^{(1)} = 3$, $\alpha_{\mathrm{Na}}^{(2)} = 12$, $\alpha_{\mathrm{Cl}}^{(1)} = 7$,
    and $\alpha_{\mathrm{Cl}}^{(2)} = 15$. For CV Set 2, there are also another alternative which has the following form,

    \begin{equation}\begin{aligned}
        n_+^{(i)} &= \sum_{j=1}^{N_{wat}} \frac{1-\left(\frac{\left\lVert\mathbf{r}_{\mathrm{Na-O}_j}\right\rVert}{b_{\mathrm{Na}}^{(i)}}\right)^n}
            {1-\left(\frac{\left\lVert\mathbf{r}_{\mathrm{Na-O}_j}\right\rVert}{b_{\mathrm{Na}}^{(i)}}\right)^m} \\
        n_-^{(i)} &= \sum_{j=1}^{N_{wat}} \frac{1-\left(\frac{\left\lVert\mathbf{r}_{\mathrm{Cl-H}_j}\right\rVert}{b_{\mathrm{Cl}}^{(i)}}\right)^n}
            {1-\left(\frac{\left\lVert\mathbf{r}_{\mathrm{Cl-H}_j}\right\rVert}{b_{\mathrm{Cl}}^{(i)}}\right)^m} \\
    \end{aligned}\end{equation}

    \noindent where $n << m$. The derivatives of these variables are also considered to be belong to this class as well.
\end{spacing}
