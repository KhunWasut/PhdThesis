\begin{spacing}{2.0}
    \chapter{GPR Computation of Two-dimensional Free Energy Landscapes of \ce{NaCl} in Aqueous Solutions}

    \section{Introduction}

    Chapters 2 and 3 took advantages of the Markovian properties of MD simulations to interpret many aspects of the dynamical information of a 
    chemical reaction; however, as shown in chapter 3, in order to get a full mechanistic picture, a good free energy landscape greatly assists
    the analysis of the metastable states and the kinetic models obtained from MSM. However, obtaining a free energy landscape with a good
    quantitative result in the barrier region is no easy task, as that implies adequate sampling of the rare event regions. However, the free
    energy computation method used in chapter 3, where the histograms of the distribution in the variable space obtained directly from unbiased
    simulations are often noisy and lack adequate resolution. Moreover, the relative free energy in the barrier region computed using this method
    have very high statistical uncertainties, since the MD trajectory would rarely visit this region. In some cases, the histogram registers zero
    visit in some particular bin, causing a gap in the free energy data, as shown in figure XX.

    A good free energy landscape with acceptable statistical uncertainties would greatly enhance the results from TICA and MSM by helping us 
    calculating a good rate constant across the free energy barrier, thereby giving a complete information of the mechanisms and the rate of any
    chemical reactions. While during earlier attempts, the free energy landscapes were often computed in the one--dimensional space due to 
    relative simplicity, as we have demonstrated in chapter 3, it is possible for a chemical reaction to contain more than one relevant reaction
    coordinates to adequately describe the reaction. Moreover, each reaction coordinate may be described by more than one dominant collective
    variables. Consequently, an efficient and quantitively robust method to compute multidimensional free energy surfaces is highly beneficial.

    This chapter presents an application of Gaussian Process Regression (GPR) for a calculation of a two--dimensional free energy landscapes
    for the reaction in \ce{NaCl + 495H2O} system that we studied in chapter 3. Here, we propose a protocol for multidimensional free energy
    surfaces calculation by combining the efficiency of fast sampling with Well-Tempered Metadynamics (WT-MTD) simulation with the GPR 
    reconstruction of the free energy landscapes from noisy free energy estimators to recreate smooth results. The quantitative agreement
    of the GPR--constructed free energy landscapes is then verified by the Eigenvector Method for Umbrella Sampling (EMUS), where a statistical
    error of the free energy landscapes in each dimension can be easily computed using one--dimensional umbrella sampling (US) simulations.
    The error of a two--dimensional free energy landscape can thus be compared efficiently with reference free energy landscapes in each dimension
    through the projection of the two--dimensional result into a one--dimensional result without the need to perform an expensive US reference
    simulation in two dimensions.
\end{spacing}
