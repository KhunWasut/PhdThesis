% BEGINNING OF THE MAIN MATTER PART OF THE THESIS
\mainmatter

% DEFINE PAGE NUMBERING - SIMILAR SCHEME TO THE ONE IN FRONT MATTER
\pagestyle{fancy}
\fancyhf{}
\renewcommand{\headrulewidth}{0pt}
\fancyfoot[CE,CO]{\thepage}

% DEFINE HOW WE WANT TO SET THE CHAPTER
\titleformat{\chapter}[block]{\centering\bfseries\large}{\centering\Large CHAPTER \thechapter \\}{0em}{}
\titleformat{\section}[block]{\bfseries\large}{\thesection }{1.5em}{}
\titlespacing*{\chapter}{0pt}{-2.5em}{0pt}
\titlespacing*{\section}{0pt}{0pt}{0pt}

\begingroup
%\renewcommand{\clearpage}{}
\chapter{Historical Backgrounds of Ion Pairing Studies in Aqueous Solutions}
\endgroup

\begin{spacing}{2.0}

    \section{Motivations}

    % SYNCED WITH LYX CHAPTER 1 DRAFT - 20180227 VERSION
    Despite being a simple chemical system, studying ion pairing in aqueous solutions could be the key to understanding many other complex 
    chemical reactions of current research interests, for the prescence of the ions in aqueous solution is an integral part of human body, 
    seawater, catalytic environments, or energy-efficient materials. There are numerous evidences of the ions in aqueous solutions that play 
    crucial roles in complex biological processes, such as the effects on macrobiomolecules like proteins or DNA, such as salt bridge formations 
    in proteins, protein-DNA interactions, or assisting the creation of tertiary and quarternary structures of proteins, [CITE MINOR] Moreover, 
    the ions in an oceanic environment also regulates the chemistry of both the ocean and the atmosphere, which could provide a better 
    mechanistic understanding of global warming. [CITE MINOR] Recent advances of batteries and energy materials also necessitate a good 
    understanding in ion transport in aqueous solution. [CITE MINOR]

    With water molecules removed, the ion pairing interactions is easily characterized as mostly electrostatic interactions between the cation 
    and the anion, which does not have any associated barriers due to the inverse dependence on the interionic separation, $r_{+-}$. However, ions 
    in aqueous solutions behave differently, as polar water molecules could arrange around the ions by the means of charge - dipole interactions, 
    solvating the ions. Thus, the solvation structure around the ions need to change in order for the cation and the anion to come close. The need 
    to overcome the solvation energy of the ions to put two ions closer together; hence, involves a free energy barrier. For this reason, 
    thermodynamics and kinetics information of ionic association in aqueous solutions can be obtained given that the free energy landscape of this 
    chemical process can be extracted, paving the ways to isolate the most likely solvation structures around the ions and elucidate the mechanisms 
    of the reaction in terms of the proper reaction coordinate for this system.

    The biggest obstacle for thoroughly understanding not only this kind of process, but also the thermodynamics and the kinetics of all kinds of 
    chemical reactions, is the lack of knowledge of such a reaction coordinate. Throughout the document, the reaction coordinate is defined as a 
    linear combination of functions $\xi_i(\mathbf{x})$, where each $\xi_i$ is called a collective variable. A collective variable is a function 
    of all the Cartesian coordinate in the phase space defining a collective group of atoms in the regions of our interests, which includes, but 
    not limited to $r_{+-}$, the angle, the dihedral angle, or even the coordination number around an atom.

    \begin{equation}
        \Psi = \sum_i c_i \xi_i(\mathbf{x})
    \end{equation}

    The definition of a proper reaction coordinate $\Psi$ for a chemical reaction is such coordinate should capture the slowest motion [CHECK] 
    across the free energy barrier in the direction of the isocommitting surface. However, in most cases, we do not know a proper combination of 
    the collective variables, nor that we know which collective variables should we choose. Therefore, most of the attempts to study the mechanisms 
    of chemical reactions usually based on the intuitive guesses of a few collective variables. As a result, we cannot know with certainty about 
    the true underlying mechanism that actually represents the slowest motion, for we only have a limited picture of all the influences.

    %The efforts of characterizing such reaction dated back the 1950s, where Sadek and Fuoss \cite{P-JACS-1954-v76-Sadek} and Winstein et al. 
    %\cite{P-JACS-1956-v78-Winstein} proposed that there are two prominent forms of ion pairing – contacted ion pair (CIP), and solvent-separated 
    %ion pairing (SSIP). The main difference between CIP and SSIP is the number of solvent molecules shared between the cation and the anion.

    %\begin{equation}
    %    \underset{\text{free ions}}{\mathrm{M}^+ + \mathrm{X}^-\vphantom{p}} \iff 
    %    \underset{\text{SSIP}}{\mathrm{M} \cdots \mathrm{X}\vphantom{p}} \iff 
    %    \underset{\text{CIP}}{\mathrm{M} \cdot \mathrm{X}\vphantom{p}}
    %    \label{eq:c1s1-ionpairillus}
    %\end{equation}

    %As outlined in \ref{eq:c1s1-ionpairillus}, the cation and the anion are directly contacted in the CIP structure; hence, there should be
    %no solvent molecules in between the two ions, and the interionic separations of the ions should be shorter. In contrast, the SSIP structure
    %has the solvent molecules acting as a bridge between the cation and the anion, hence the longer separation between the two ions. Therefore,
    %in order for the ions to be able to come close, the hydration shells around these ions need to rearrange to make room for the ions to
    %be come into a closer contact.

\end{spacing}
