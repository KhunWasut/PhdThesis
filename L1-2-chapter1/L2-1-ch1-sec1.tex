% BEGINNING OF THE MAIN MATTER PART OF THE THESIS
\mainmatter

% DEFINE PAGE NUMBERING - SIMILAR SCHEME TO THE ONE IN FRONT MATTER
\pagestyle{fancy}
\fancyhf{}
\renewcommand{\headrulewidth}{0pt}
\fancyfoot[CE,CO]{\thepage}

% DEFINE HOW WE WANT TO SET THE CHAPTER
\titleformat{\chapter}[block]{\centering\bfseries\large}{\centering\Large CHAPTER \thechapter \\}{0em}{}
\titleformat{\section}[block]{\bfseries\large}{\thesection }{1.5em}{}
\titlespacing*{\chapter}{0pt}{-2.5em}{0pt}
\titlespacing*{\section}{0pt}{0pt}{0pt}

\begingroup
%\renewcommand{\clearpage}{}
\chapter{Ion Association in Aqueous Solutions - Introduction}
\endgroup

\begin{spacing}{2.0}

    \section{Background of the Problem}

    % CURRENT TEXT - SYNC WITH WORD VERSION!! TO APPLY NEW CHANGES, COPY THE WHOLE THING FROM WORD AND KEEP OLD TEXT FIRST
    % AS COMMENTS FOR COMPARISON!!
    Ionic association in aqueous solution is a chemical process defined by the interactions between the cation and the anion as a solute of an 
    aqueous solution. \textbf{MAYBE ADD THE IMPORTANCE OF THIS CHEMICAL PROCESS HERE...} Due to the charged nature of the ions and the fact that 
    water molecule is a highly polar chemical species, water molecules have a strong tendency to solvate around the ions, forming a hydration 
    shell. The formation of a hydration shell can be thought of as a cage, where an ion in the hydration shell is confined in such a cage by 
    the charge – dipole interaction. Therefore, hydrated ions become less effective in chemical interactions in an aqueous solution environment 
    because there is an energy cost associated with breaking the ion-solvent interactions. Hence, understanding the behavior of the solvent is 
    crucial.

    The efforts of characterizing such reaction dated back the 1950s, where Sadek and Fuoss \cite{P-JACS-1954-v76-Sadek} and Winstein et al. 
    \cite{P-JACS-1956-v78-Winstein} proposed that there are two prominent forms of ion pairing – contacted ion pair (CIP), and solvent-separated 
    ion pairing (SSIP). The main difference between CIP and SSIP is the number of solvent molecules shared between the cation and the anion.

    \begin{equation}
        \underset{\text{free ions}}{\mathrm{M}^+ + \mathrm{X}^-\vphantom{p}} \iff 
        \underset{\text{SSIP}}{\mathrm{M} \cdots \mathrm{X}\vphantom{p}} \iff 
        \underset{\text{CIP}}{\mathrm{M} \cdot \mathrm{X}\vphantom{p}}
        \label{eq:c1s1-ionpairillus}
    \end{equation}

    As outlined in \ref{eq:c1s1-ionpairillus}, the cation and the anion are directly contacted in the CIP structure; hence, there should be
    no solvent molecules in between the two ions, and the interionic separations of the ions should be shorter. In contrast, the SSIP structure
    has the solvent molecules acting as a bridge between the cation and the anion, hence the longer separation between the two ions. Therefore,
    in order for the ions to be able to come close, the hydration shells around these ions need to rearrange to make room for the ions to
    be come into a closer contact.

\end{spacing}
