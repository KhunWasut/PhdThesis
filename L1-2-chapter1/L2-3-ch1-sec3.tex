\section{Areas of Improvement and Objectives}
\label{sec:L12L23}

\begin{spacing}{2.0}

    The work of Mullen et al. changed the perception of how we should approach this problem in a hugely significant manner by introducing a number 
    of possible candidates to describe the dynamics of the solvent; however, there are still several obstacles for us to fully grasp the whole 
    picture. There are two possible questions that arose from Mullen et al. First, although the proper reaction coordinate can be found by maximizing 
    the likelihood of a linear combination of collective variables that go across the dividing surface between two isocommittor surfaces, there is 
    a possibility that more than 3 collective variables are actually involved and a linear combination of 3 collective variables may still be 
    inadequate to properly representing the slowest motion. Second, the work of Mullen et al. focused mainly in the SSIP - CIP transition, but did 
    not discuss the dynamics of the bulk, so the mechanics of the ionic association from the bulk remain relatively little understood.

    There are several ways that can be used to identify proper reaction coordinates for a chemical reaction. The maximum likelihood method used in 
    Mullen et al. is just one method among many other proposals of methods that allow us to study the most important coordinate of a chemical system, 
    such as string method \cite{P-PhysRevB-2002-v66-E, P-JPhysChemB-2005-v109-E, P-JChemPhys-2007-v126-E}, markov state model (MSM)
    \cite{P-JChemPhys-2004-v121-Singhal, P-JChemPhys-2008-v129-Pan, P-Methods-2010-v52-Pande, P-PhysChemChemPhys-2011-v13-Prinz, P-JChemPhys-2011-v134-Prinz}, 
    principal component analysis (PCA), diffusion map, and many other. \cite{P-AnnuRevPhysChem-2013-v64-Rohrdanz} [CITE RELATED] Among these methods, 
    MSM is one of the most widely used tools in biochemical simulations to identify different conformations of large biomolecules such as proteins or DNA. 
    \cite{P-JCTC-2014-v10-Malmstrom} [CITE MSM WORKS] Besides being used to identify metastable states in the system, the information from MSM 
    eigenfunctions also directly relates to different motions in a chemical system. Therefore, not only that MSM could give an insight for the slowest 
    motion, MSM can also be useful in a stituation where a system with motions in a very similar timescale and the slowest motion cannot be fully 
    distinguished from the next slowest motion in the system. This would allow us to study any possible secondary or tertiary motions and their possible 
    effects on the primary motion across the free energy barrier. Besides, the information of each MSM eigenfunction can also be projected onto the 
    collective variables space and we could determine each unique linear combination of the collective variables specific to each motion in the system.
    \cite{P-JChemPhys-2013-v139-Perez-Hernandez, P-JChemPhys-2017-v146-Wu, P-CurrOpStructBiol-2017-v43-Noe}

    In accompanying a proper reaction coordinate which may have contributions from several collective variables, a multidimensional free energy 
    landscape is needed to represent those collective variables to give a viable insight into the reaction mechanisms. However, the current capabilities 
    limited the free energy computation to a two-dimensional space due to the high computational cost of multidimensional sampling. As relative 
    free energy between two points in the collective variable space relates to the natural logarithm of the probability of going from one state to 
    another, any free energy barriers are considered as rare event regions that are naturally less sampled for any unbiased molecular dynamics 
    simulations. Hence, we have relatively little information to correctly determine the free energy landscapes around the barriers, which may 
    cause errors in rate calculation and inaccurate description of the transition state - a crucial element for full understandings of the entire 
    mechanism. For this type of problem, Cuendet and Tuckerman has noted the difficulties in the free energy computation of the ionic association 
    reaction of NaCl in aqueous solution which presents challenges for any novel free energy computation methods that usually validate their models 
    with the free energy landscape of alanine dipeptide with respect to their two dihedral angles. Compared to the free energy of alanine dipeptide, 
    the free energy of NaCl's SSIP and CIP states differ only in the order of a few kcal/mol, prompting the need to make sure that we need a free 
    energy calculation scheme that cannot produce a large error which would otherwise cause severe inaccuracies in the result. Moreover, the CIP 
    feature of NaCl's one-dimensional free energy is relatively narrow, which implies that the free energy gradient is relatively rapidly changing 
    and giving rise to statistical noise when averaging the free energy estimators. \cite{P-JCTC-2014-v10-Cuendet}

    The preferred method of free energy computation among the community has always been window-based simulations, where an average of a probability 
    density or average force along a particular collective variable was computed in a restrained simulation environment to ensure adequate amount of 
    sampling, especially in the rare event regions. Despite providing accurate results, the main disadvantage of this class of free energy simulation 
    method has always been the cost, which scales as $\mathcal{O}(n^D)$ for a $D$-dimensional problem. Although it is also possible to perform a 
    simulation in each window in parallel given that we know a good initial configuration in each window, this does not eliminate the scaling issue 
    for a many-dimension problem. \cite{P-JCTC-2013-v9-Wojtas-Niziurski} Another main issue in free energy simulation is the inherent statistical 
    noises of the average probability density or an average force along a collective variable, which also produce noisy free energy landscapes that 
    may become an issue for a chemical system with sensitive free energy profile. 

    In order to circumvent the two issues of free energy computations presented in the above paragraph, we need to first be able to quickly explore 
    the free energy landscape to eliminate the need to perform windowed simulation while also collecting the needed information to compute the free 
    energy landscapes, and then we need to find a reconstruction algorithm that ensures a smooth reconstructed data. The first task could be done 
    using well-tempered metadynamics (WT-MTD) simulation, which has been shown in the work by Mones et al. to have fastest exploration of the free 
    energy landscape compared to many other algorithms. \cite{P-JCTC-2016-v12-Mones} During the WT-MTD simulation, biased instantaneous forces (BIFs) 
    along the collective variables are collected and then later are unbiased using the known information of the deposited WT-MTD Gaussians to obtain 
    unbiased instantaneous forces (UIFs). The UIFs could be treated as a noisy, unaveraged version of the average force along the collective variable 
    commonly used in thermodynamic integration to compute the free energy landscapes. Therefore, a machine learning-based approach such as Gaussian 
    process regression (GPR) could be used to infer the most likely free energy landscape based on our training data of UIFs, providing a smooth 
    reconstruction of the free energy landscape in any number of dimensions. \cite{P-JCTC-2014-v10-Stecher, P-JCTC-2016-v12-Mones}

\end{spacing}
